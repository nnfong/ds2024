\documentclass[12pt]{article}
\usepackage{amsmath}
\usepackage{graphicx}
\usepackage{hyperref}
\usepackage[latin1]{inputenc}
\usepackage[top=2cm, bottom=2cm, left=2cm, right=2cm, headsep=14pt]{geometry}
\usepackage[T1]{fontenc}
\usepackage[utf8]{inputenc}

\title{Report 2}
\author{Phong}
%
\date{04/21/2024}

\begin{document}
\maketitle

\section{Designing RPC service}
I use RPCGEN to generate stubs and runtime program for communication between client and server. When the client initiates a message, the stub included in the client code receives the request and sends it to the client runtime program, which will address the server and then send a message to the server to execute the remote procedure.

\begin{figure}[h]
    \centering
    \includegraphics[scale=0.4]{RPC_Service.png}
    \caption{RPC service}
\end{figure}
  
\section{Organizing the system}
In the client, we write functions to read files into a buffer, and we compile it together with the RPCGEN code, which allows data to be transferred from the client to the server and makes a remote procedure call. In the server, so i write functions to write data from the buffer and compile it with the RPCGEN code, which allows us to receive requests from clients and execute functions, also create a sample to store structs.

\begin{figure}[h]
    \centering
    \includegraphics[height=8.35cm]{RPC_organize.png}
    \caption{System}
\end{figure}

\section{Implementing the file transfer}
In the client, I write functions to open files, create a buffer, and read files. The buffer has three attributes: one to store the file name, one to store the data, and one to store the number of bytes. The client then calls the remote procedure and sends the entire buffer to the server.

\begin{verbatim}
while (1) {
    /* Read a BLOCK of bytes from file to data. */
    read_bytes = fread(buff.data, 1, BLOCK, file);
    total_bytes += read_bytes;
    buff.numbytes = read_bytes;
    result = filetransfer_proc_1(&buff, cl);
}
\end{verbatim}

A result will show if the process works or not.

\begin{verbatim}
if (result == NULL) {
    /* An RPC error occurred while calling the server. */
    clnt_perror(cl, host);
    exit(1);
}
/* Successfully called the remote procedure. */
if (*result == 0) {
    /* A remote system error occurred. */
    errno = *result;
    perror(name);
    exit(1);
}
\end{verbatim}

File data will be read by blocks of 1024 bytes until all the data on the server.

After the client receives the remote procedure call from the client, the server will execute the function to write the file (export file) through the buffer.

\begin{verbatim}
file = fopen(recvbuff->name, "a");
if (file == NULL) {
    write_bytes = -1;
    return &write_bytes;
}
write_bytes = fwrite(recvbuff->data, 1, recvbuff->numbytes, file);
fclose(file);
\end{verbatim}
 
\end{document}
